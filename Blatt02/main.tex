\documentclass[DIN, pagenumber=false, fontsize=11pt, parskip=half]{scrartcl}

\usepackage{amsmath}
\usepackage{amsfonts}
\usepackage{amssymb}
\usepackage{enumitem}
\usepackage[utf8]{inputenc} 
\usepackage[ngerman]{babel} 
\usepackage[T1]{fontenc} 
\usepackage{pgfplots}
\usepackage{xcolor}
\usepackage{listings}
\usepackage{float}
\usepackage{graphicx}
\usepackage{booktabs}

\definecolor{mygreen}{RGB}{28,172,0} % color values Red, Green, Blue
\definecolor{mylilas}{RGB}{170,55,241}

\lstset{language=Matlab,%
    %basicstyle=\color{red},
    breaklines=true,%
    morekeywords={matlab2tikz},
    keywordstyle=\color{blue},%
    morekeywords=[2]{1}, keywordstyle=[2]{\color{black}},
    identifierstyle=\color{black},%
    stringstyle=\color{mylilas},
    commentstyle=\color{mygreen},%
    showstringspaces=false,%without this there will be a symbol in the places where there is a space
    numbers=left,%
    numberstyle={\tiny \color{black}},% size of the numbers
    numbersep=9pt, % this defines how far the numbers are from the text
    emph=[1]{for,end,break},emphstyle=[1]\color{red}, %some words to emphasise
    %emph=[2]{word1,word2}, emphstyle=[2]{style},    
}

\title{Einführung in die Neuroinformatik}
\author{Tim Luchterhand, Paul Nykiel}

\begin{document}
    \maketitle
    \section{Boole'sche Funktionen}
    \subsection{Logische Funktionen}
    \begin{enumerate}[label = (\alph*)]
        \item AND-Gatter
            \begin{eqnarray*}
                w &=& \begin{pmatrix}
                    1 \\ 1 \\ 1
                \end{pmatrix} \\
                \theta &=& 3
            \end{eqnarray*}
        \item OR-Gatter
            \begin{eqnarray*}
                w &=& \begin{pmatrix}
                    1 \\ 1 \\ 1
                \end{pmatrix} \\
                \theta &=& 1
            \end{eqnarray*}
    \end{enumerate}
    \subsection{Eindeutigkeit}
    Nein, da der Vektor $w$ und $\theta$ zum Beispiel jeweils mit einem Skalar skaliert werden können.
    \subsection{Beispiel}
    \begin{enumerate}[label=(\alph*)]
        \item @TODO Tikz
        \item 
            \begin{eqnarray*}
                w &=& \begin{pmatrix}
                    2 \\ 1 \\ 1
                \end{pmatrix} \\
                \theta &=& 3
            \end{eqnarray*}
    \end{enumerate}
    \section{Schwellwertneuronen}
    \subsection{Wahrscheinlickeiten}
    @TODO Tikz
    \subsection{Ebene im $\mathbb{R}^3$}
    \begin{eqnarray*}
        x_1 + x_2 + x_3 &\stackrel{!}{=}& 1 \\ 
        \Leftrightarrow x_1 + x_2 + x_3 \stackrel{!}{\geq} 1 &\land& x_1 + x_2 + x_3 \stackrel{!}{\leq} 1\\
        \Leftrightarrow x_1 + x_2 + x_3 \stackrel{!}{\geq} 1 &\land&  - x_1 - x_2 - x_3 \stackrel{!}{\geq} -1\\
    \end{eqnarray*}
    TODO Tikz
    \section{Logistisches Neuron}
    \subsection{}
    \begin{enumerate}[label=(\alph*)]
        \item Eine Erhöhung von $w_1$ vergrößert die Steigung von $y_1(x)$
        \item Bei negativem $w_1$ fällt $y_1(x)$ von $1$ auf $0$ ab, bei positivem $w_1$ steigt sie von $0$ auf $1$ an.
        \item Einfluss von Bias und Gewicht:
            \begin{table}[H]
                \centering
                \begin{tabular}{ccc}
                    \toprule
                     & $w_1$ > 0 & $w_1 < 0$ \\
                     \midrule
                     $b_1 \uparrow $ & Rechts & Rechts\\
                     $b_1 \downarrow$ & Links & Links\\
                     \bottomrule
                \end{tabular}
            \end{table}
    \end{enumerate}
    \subsection{}
    Durch Ausprobieren:
    \begin{eqnarray*}
        b_1 &=& b_2 = -1 \\
        w_1 &=& -w_2 = 1 \\
    \end{eqnarray*}
\end{document}
