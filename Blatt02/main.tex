\documentclass[DIN, pagenumber=false, fontsize=11pt, parskip=half]{scrartcl}

\usepackage{amsmath}
\usepackage{amsfonts}
\usepackage{amssymb}
\usepackage{enumitem}
\usepackage[utf8]{inputenc} 
\usepackage[ngerman]{babel} 
\usepackage[T1]{fontenc} 
\usepackage{pgfplots}
\usepackage{xcolor}
\usepackage{listings}
\usepackage{float}
\usepackage{graphicx}
\usepackage{booktabs}

\definecolor{mygreen}{RGB}{28,172,0} % color values Red, Green, Blue
\definecolor{mylilas}{RGB}{170,55,241}

\lstset{language=Matlab,%
    %basicstyle=\color{red},
    breaklines=true,%
    morekeywords={matlab2tikz},
    keywordstyle=\color{blue},%
    morekeywords=[2]{1}, keywordstyle=[2]{\color{black}},
    identifierstyle=\color{black},%
    stringstyle=\color{mylilas},
    commentstyle=\color{mygreen},%
    showstringspaces=false,%without this there will be a symbol in the places where there is a space
    numbers=left,%
    numberstyle={\tiny \color{black}},% size of the numbers
    numbersep=9pt, % this defines how far the numbers are from the text
    emph=[1]{for,end,break},emphstyle=[1]\color{red}, %some words to emphasise
    %emph=[2]{word1,word2}, emphstyle=[2]{style},    
}

\title{Einführung in die Neuroinformatik}
\author{Tim Luchterhand, Paul Nykiel}

\begin{document}
    \maketitle
    \section{Boole'sche Funktionen}
    \subsection{}
    \begin{enumerate}[label = (\alph*)]
        \item AND-Gatter
            \begin{eqnarray*}
                w &=& \begin{pmatrix}
                    1 \\ 1 \\ 1
                \end{pmatrix} \\
                \theta &=& 3
            \end{eqnarray*}
        \item OR-Gatter
            \begin{eqnarray*}
                w &=& \begin{pmatrix}
                    1 \\ 1 \\ 1
                \end{pmatrix} \\
                \theta &=& 1
            \end{eqnarray*}
    \end{enumerate}
    \subsection{}
    Nein, da der Vektor $w$ und $\theta$ zum Beispiel jeweils mit einem Skalar skaliert werden können.
    \subsection{}
    \begin{enumerate}[label=(\alph*)]
        \item $ $
            \begin{figure}[H]
                \centering
                \begin{tikzpicture}
                    \tikzstyle{neuron}=[circle,fill=black!25,minimum size=24pt,inner sep=0pt]

                    \node[anchor=east] (Wetter) at (0,8) {Wetter};
                    \node[anchor=east] (Begleitung) at (0,4) {Begleitung};
                    \node[anchor=east] (Essen) at (0,0) {Essen};

                    \node[neuron] (WetterFilter) at (4,6) {$\geq 1$};
                    \node[neuron] (Oder) at (4,2) {$\geq 1$};

                    \node[neuron, pin=right:Output] (Und) at (8,4) {$\geq 2$};

                    \path (Wetter) edge[->] node[anchor=south,pos = 0.8]{$1$} (WetterFilter);
                    \path (Wetter) edge[->] node[anchor=south, pos=0.8]{$0$} (Oder);

                    \path (Begleitung) edge[->] node[anchor=south, pos=0.8]{$1$} (Oder);
                    \path (Essen) edge[->] node[anchor=south, pos=0.8]{$1$} (Oder);

                    \path (Begleitung) edge[->] node[anchor=south, pos=0.8]{$0$} (WetterFilter);
                    \path (Essen) edge[->] node[anchor=south, pos=0.8]{$0$} (WetterFilter);

                    \path (WetterFilter) edge[->] node[anchor=south, pos=0.8]{$1$} (Und);
                    \path (Oder) edge[->] node[anchor=south, pos=0.8]{$1$} (Und);
                \end{tikzpicture}
            \end{figure}
        \item 
            \begin{eqnarray*}
                w &=& \begin{pmatrix}
                    2 \\ 1 \\ 1
                \end{pmatrix} \\
                \theta &=& 3
            \end{eqnarray*}
    \end{enumerate}
    \section{Schwellwertneuronen}
    \subsection{}
    \begin{figure}[H]
        \centering
        \begin{tikzpicture}
            \tikzstyle{neuron}=[circle,fill=black!25,minimum size=24pt,inner sep=0pt]

            \node[anchor=east] (Wetter) at (0,8) {Wetter};
            \node[anchor=east] (Begleitung) at (0,4) {Begleitung};
            \node[anchor=east] (Essen) at (0,0) {Essen};

            \node[neuron] (WetterFilter) at (4,8) {$\geq 0.5$};
            \node[neuron] (BegleitungFilter) at (4,4) {$\geq 0.25$};
            \node[neuron] (EssenFilter) at (4,0) {$\geq 0.25$};

            \node[neuron, pin=right:Output] (Output) at (8,4) {$\geq 3$};

            \path (Wetter) edge[->] node[anchor=south,pos = 0.8]{$1$} (WetterFilter);
            \path (Wetter) edge[->] node[anchor=south, pos=0.8]{$0$} (BegleitungFilter);
            \path (Wetter) edge[->] node[anchor=south, pos=0.8]{$0$} (EssenFilter);

            \path (Begleitung) edge[->] node[anchor=south,pos = 0.8]{$0$} (WetterFilter);
            \path (Begleitung) edge[->] node[anchor=south, pos=0.8]{$1$} (BegleitungFilter);
            \path (Begleitung) edge[->] node[anchor=south, pos=0.8]{$0$} (EssenFilter);

            \path (Essen) edge[->] node[anchor=south,pos = 0.8]{$0$} (WetterFilter);
            \path (Essen) edge[->] node[anchor=south, pos=0.8]{$0$} (BegleitungFilter);
            \path (Essen) edge[->] node[anchor=south, pos=0.8]{$1$} (EssenFilter);

            \path (WetterFilter) edge[->] node[anchor=south, pos=0.8]{$2$} (Output);
            \path (BegleitungFilter) edge[->] node[anchor=south, pos=0.8]{$1$} (Output);
            \path (EssenFilter) edge[->] node[anchor=south, pos=0.8]{$1$} (Output);
        \end{tikzpicture}
    \end{figure}
    \subsection{}
    \begin{eqnarray*}
        x_1 + x_2 + x_3 &\stackrel{!}{=}& 1 \\ 
        \Leftrightarrow x_1 + x_2 + x_3 \stackrel{!}{\geq} 1 &\land& x_1 + x_2 + x_3 \stackrel{!}{\leq} 1\\
        \Leftrightarrow x_1 + x_2 + x_3 \stackrel{!}{\geq} 1 &\land&  - x_1 - x_2 - x_3 \stackrel{!}{\geq} -1\\
    \end{eqnarray*}
    Daraus folgt für das Netz:
    \begin{figure}[H]
        \centering
        \begin{tikzpicture}
            \tikzstyle{neuron}=[circle,fill=black!25,minimum size=24pt,inner sep=0pt]

            \node[anchor=east] (x1) at (0,8) {$x_1$};
            \node[anchor=east] (x2) at (0,4) {$x_2$};
            \node[anchor=east] (x3) at (0,0) {$x_3$};

            \node[neuron] (N1) at (4,6) {$\geq 1$};
            \node[neuron] (N2) at (4,2) {$\geq -1$};

            \node[neuron, pin=right:Output] (Output) at (8,4) {$\geq 2$};

            \path (x1) edge[->] node[anchor=south,pos = 0.8]{$1$} (N1);
            \path (x1) edge[->] node[anchor=south,pos = 0.8]{$-1$} (N2);

            \path (x2) edge[->] node[anchor=south,pos = 0.8]{$1$} (N1);
            \path (x2) edge[->] node[anchor=south,pos = 0.8]{$-1$} (N2);

            \path (x3) edge[->] node[anchor=south,pos = 0.8]{$1$} (N1);
            \path (x3) edge[->] node[anchor=south,pos = 0.8]{$-1$} (N2);

            \path (N1) edge[->] node[anchor=south,pos = 0.8]{$1$} (Output);
            \path (N2) edge[->] node[anchor=south,pos = 0.8]{$1$} (Output);
        \end{tikzpicture}
    \end{figure}
    \section{Logistisches Neuron}
    \subsection{}
    \begin{enumerate}[label=(\alph*)]
        \item Eine Erhöhung von $w_1$ vergrößert die Steigung von $y_1(x)$
        \item Bei negativem $w_1$ fällt $y_1(x)$ von $1$ auf $0$ ab, bei positivem $w_1$ steigt sie von $0$ auf $1$ an.
        \item Einfluss von Bias und Gewicht:
            \begin{table}[H]
                \centering
                \begin{tabular}{ccc}
                    \toprule
                     & $w_1$ > 0 & $w_1 < 0$ \\
                     \midrule
                     $b_1 \uparrow $ & Rechts & Rechts\\
                     $b_1 \downarrow$ & Links & Links\\
                     \bottomrule
                \end{tabular}
            \end{table}
    \end{enumerate}
    \subsection{}
    Durch Ausprobieren:
    \begin{eqnarray*}
        b_1 &=& b_2 = -1 \\
        w_1 &=& -w_2 = 1 \\
    \end{eqnarray*}
\end{document}
