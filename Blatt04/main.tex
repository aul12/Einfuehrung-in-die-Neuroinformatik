\documentclass[DIN, pagenumber=false, fontsize=11pt, parskip=half]{scrartcl}

\usepackage{amsmath}
\usepackage{amsfonts}
\usepackage{amssymb}
\usepackage{enumitem}
\usepackage[utf8]{inputenc} 
\usepackage[ngerman]{babel} 
\usepackage[T1]{fontenc} 
\usepackage{pgfplots}
\usepackage{xcolor}
\usepackage{listings}
\usepackage{float}
\usepackage{graphicx}
\usepackage{booktabs}

\definecolor{mygreen}{RGB}{28,172,0} % color values Red, Green, Blue
\definecolor{mylilas}{RGB}{170,55,241}

\tikzstyle{neuron}=[circle,fill=black!25,minimum size=30pt,inner sep=0pt]

\lstset{language=Matlab,%
    %basicstyle=\color{red},
    breaklines=true,%
    morekeywords={matlab2tikz},
    keywordstyle=\color{blue},%
    morekeywords=[2]{1}, keywordstyle=[2]{\color{black}},
    identifierstyle=\color{black},%
    stringstyle=\color{mylilas},
    commentstyle=\color{mygreen},%
    showstringspaces=false,%without this there will be a symbol in the places where there is a space
    numbers=left,%
    numberstyle={\tiny \color{black}},% size of the numbers
    numbersep=9pt, % this defines how far the numbers are from the text
    emph=[1]{for,end,break},emphstyle=[1]\color{red}, %some words to emphasise
    %emph=[2]{word1,word2}, emphstyle=[2]{style},    
}

\title{Einführung in die Neuroinformatik}
\author{Tim Luchterhand, Paul Nykiel (Gruppe P)}

\begin{document}
    \maketitle
    \section{Lernschritt im Perzeptron-Lernalgorithmus}
    \begin{enumerate}[label=(\alph*)]
        \item TODO
        \item
            \begin{eqnarray*}
                w^* &=& \begin{pmatrix}
                    1 \\ 2\\ -1
                \end{pmatrix} \\
                x^* &=& \begin{pmatrix}
                    2 \\ 2\\ 1
                \end{pmatrix}
            \end{eqnarray*}
        \item TODO
        \item Überprüfen ob bereits korrekt klassifiziert:
            \begin{eqnarray*}
                {(w^*)}^T \cdot x^* &=& 2 + 4 - 1 = 5 \geq 0 \\
                &\Rightarrow& x^* \in \omega_{1}
            \end{eqnarray*}
            Lernschritt durchführen:
            \begin{eqnarray*}
                \tilde{w}^* &=& w^* - \nu \cdot x^* = \begin{pmatrix}
                    -1 \\ 0 \\ -2 
                \end{pmatrix} \\
                \Rightarrow \tilde{w} &=& 
                    \begin{pmatrix}
                        -1 \\ 0
                    \end{pmatrix} \\
                \tilde{x} &=& -2
            \end{eqnarray*}
        \item TODO
        \item Überprüfen ob bereits korrekt klassifiziert:
            \begin{eqnarray*}
                {(\tilde{w}^*)}^T \cdot x^*
                &=& -2 + 0 -2 = -4 \\
                &\Rightarrow& x^* \in \omega_{-1}
            \end{eqnarray*}
            Kein Lernschritt ist notwendig $\Rightarrow w$ wird nicht verändert
    \end{enumerate}
    \section{Perzeptron-Lernalgorithmus}
    Matlab script:
    \lstinputlisting{b04a02.m}
\end{document}
